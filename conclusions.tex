%!TEX root = ms2.tex

\section{Conclusions}\label{sec:conclusions}
In this paper we measure the clustering amplitude of the \Xray, \Radio, and \IR populations in the DEEP2 and PRIMUS fields.
We focus on these samples to identify a large range of specific accretion rate (Eddington ratio) sources to correlate environmental trends with respect to different accretion rates.
We measure the clustering strength of the \Xray sample relative to X-ray luminosity, specific accretion rate, hardness ratio subsamples.
We estimate the clustering strength of the radio-loud and put an upper bound on the radio-quiet samples.
We use WISE mid-IR to compare obscured and unobscured \Assef samples.
We quantify the relative clustering strengths between each sample, and  compared to stellar mass, star formation rate and redshift matched galaxy control samples.
The main results from our work are as follows:

\begin{enumerate}

\item The inferred two-point auto-correlation function of \Xray, \Radio and \Donley sources to reside in different mass dark matter halos. 
From the two-halo term ($1<\rp/\hMpc<10$) we estimate that the \Xray, \Radio, and \Donley samples have median dark matter halo masses of 
\Mhalo{\sim}{13.3}, \Mhalo{\sim}{13.6}, and \Mhalo{\sim}{11.6}, respectively.

\item The \Xray sample clusters similarly to the \Radio sample.  Both are more ($\gtrsim$50\%; $\sim$2$\sigma$) clustered than the \Donley sample. The \Xray and \Radio samples are much more clustered in the one halo term as compared to the \Donley sample.

\item We find no significant correlations ($<$2$\sigma$) between the clustering amplitude and subsamples of the \Xray population with X-ray luminosity, specific accretion rate, and hardness ratio.
Trends with respect to stellar mass, star formation rate, and redshift are well within the expected trend for inactive matched galaxy control samples.

\item We find no significant difference in the clustering amplitude of  Radio-loud and all radio detected galaxies.
Since these samples probe different median redshifts ($z\sim0.5$ vs. $z\sim0.8$), we find that the radio-quiet sample to have a larger dark matter halo mass (\Mhalo{\sim}{13.6} vs. \Mhalo{\sim}{13.9}).

\item The \Donley and \Assef samples have similar clustering amplitudes.  For either we find no significant difference in the clustering amplitude of obscured and un-obscured \Assef sources.
This suggests differences found in angular clustering results are dominated by the uncertainty of the redshift distribution.

\item From the large range of specific accretion rates that the \Xray sample probes we can explain the bulk of the differences in clustering amplitudes found in the \Xray, \Radio, and \IR samples.
The high specific-accretion rate \Xray sources cluster more similar to the \Donley sample.
The low specific-accretion rate \Xray sources cluster similar to the \Radio sample.
This suggests that the environmental dependance on the different multi-wavelength AGN samples is sub-dominate on the scales that we probe $\gtrsim100\hKpc$.

\item The stellar mass, star formation rate, and redshift matched galaxy control samples are clustered similarly for the \Xray, \Radio samples.
The \Donley sample is less clustered than the matched galaxy control sample.
This may suggest contamination of the host galaxy light overpredicting the matched mass galaxy sample mass, and thus clustering amplitude.

\end{enumerate}


%% Support and Acknoledgements
\vspace{4em}
We thank our anonymous referee for useful comments that have greatly improved
this paper. We thank Andy Goulding for reducing and extracting the DEEP2 02hr
IRAC data.

Funding for PRIMUS has been provided by NSF grants AST-0607701, 0908246,
0908442, 0908354, and NASA grant 08-ADP08-0019. ALC acknowledges support from
the Alfred P. Sloan Foundation and NSF CAREER award AST-1055081. AJM and JA
acknowledges support from NASA grant NNX12AE23G through the Astrophysics Data
Analysis Program.

We thank the CFHTLS, COSMOS, DLS, and SWIRE teams for their public data
releases and/or access to early releases. This paper includes data gathered
with the 6.5 m Magellan Telescopes located at Las Campanas Observatory, Chile.
We thank the support staff at LCO for their help during our observations, and
we acknowledge the use of community access through NOAO observing time. We use
data from the DEEP2 survey, which was supported by NSF AST grants AST00-71048,
AST00-71198, AST05-07428, AST05-07483, AST08-07630, AST08-08133. This study
makes use of data from AEGIS Survey and in particular uses data from $\galex$,
Keck, and CFHT. The AEGIS Survey was supported in part by the NSF, NASA, and
the STFC. Some of the data used for this project are from the CFHTLS public
data release, which includes observations obtained with MegaPrime/MegaCam, a
joint project of CFHT and CEA/DAPNIA, at the Canada-France-Hawaii Telescope
(CFHT) which is operated by the National Research Council (NRC) of Canada, the
Institut National des Science de l'Univers of the Centre National de la
Recherche Scientifique (CNRS) of France, and the University of Hawaii. This
work is based in part on data products produced at TERAPIX and the Canadian
Astronomy Data Centre as part of the Canada-France-Hawaii Telescope Legacy
Survey, a collaborative project of NRC and CNRS. We also thank those who have
built and operate the Chandra and XMM-Newton X-Ray Observatories. This research
has made use of the NASA/IPAC Infrared Science Archive, which is operated by
the Jet Propulsion Laboratory, California Institute of Technology, under
contract with the National Aeronautics and Space Administration.

% Based on observations obtained with MegaPrime/MegaCam, a joint
% project of CFHT and CEA/DAPNIA, at the Canada-France-Hawaii Telescope (CFHT)
% which is operated by the National Research Council (NRC) of Canada, the Institut
% National des Science de l’Universof the Centre National de la Recherche
% Scientifique (CNRS) of France, and the University of Hawaii. 
% This work is based
% in part on data products produced at TERAPIX and the Canadian Astronomy Data
% Centre as part of the Canada-France-Hawaii Telescope Legacy Survey, a
% collaborative project of NRC and CNRS.
