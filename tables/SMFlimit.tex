%!TEX root = ../main.tex
\begin{deluxetable}{cr}
\tablewidth{0pc}
\tablecolumns{2}
\tablecaption{Matched IP sample stellar mass limits\tablenotemark{a}
\label{table:SMFlimit}}
\tablehead{
%\multicolumn{1}{c}{} & \multicolumn{4}{c}{Area [deg$^2$]} & \multicolumn{6}{c}{Sample Size} \\
% \colhead{\makecell[l]{Field}} & \colhead{X-ray} & \colhead{Radio} & \colhead{IRAC} & \colhead{WISE} & \colhead{$N_\textrm{Galaxy}$\tablenotemark{a}} & \colhead{$N_\textrm{Mass}$\tablenotemark{b}} & \colhead{$N_\textrm{X-ray}$\tablenotemark{c}} & \colhead{$N_\textrm{Radio}$\tablenotemark{d}} & \colhead{$N_\textrm{Donley}$} & \colhead{$N_\textrm{Assef}$}  }
\colhead{Redshift Range} & \colhead{$\log\,(\mlim / \msun)$} }
\label{table:SMFlimit}
\startdata
0.20\textendash0.30 & 11.154 \\
0.50\textendash0.65 & 11.241 \\
0.30\textendash0.40 & 11.208 \\
0.65\textendash0.80 & 11.308 \\
0.40\textendash0.50 & 11.255 \\
0.80\textendash1.00 & 11.324 \\
%\hline \\[-2ex]
%Totals: & 8.35 & 8.03 & 8.11 & 10.69 & 123,819 & 120,550 & 953 & 256 (971) & 442 & 237\\
\enddata
\tablenotetext{a}{$\mlim$ is the stellar mass in each redshift range where ${\log\Phi=-3.7}$.}
\end{deluxetable}