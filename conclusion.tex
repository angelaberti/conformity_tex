%!TEX root = main.tex

\section{Conclusion}\label{sec:conclusion}

Confirmation of galactic conformity at intermediate redshift is a powerful tool for constraining how halo occupation models should move beyond the standard HOD model.
While, measurements of conformity at $z>0.2$ are possible with photometric redshifts, but the large uncertainties of such measurements call their usefulness into question.

We have tested for one- and two-halo galactic conformity at ${0.2<z<1}$ with a 5.5~$\degsq$ sample of $\sim$60,000 ${\mstar\gtrsim10^{9.3}~\msun}$ galaxies from PRIMUS, the largest existing spectroscopic redshift survey of faint galaxies to ${z\sim1}$.
With four spatially distinct fields, our sample allows us to account for the effect of cosmic variance on the conformity signal, which we have shown can vary substantially among fields.
We detect a 5.3\% one-halo conformity signal at $3.6\sigma$, and a 1.1\% two-halo term at $2.5\sigma$.

Ours is the largest area intermediate redshift conformity study to date, and the only measurement of conformity at $z>0.2$ using spectroscopic redshifts.
While our detections are robust, the fact that a survey the size of PRIMUS is only large enough to detect a $\sim$1\% two-halo conformity signal at the $2.5\sigma$ level illustrates the need for a next generation of deep, wide-field spectroscopic redshift surveys beyond the ${z<0.2}$ range of SDSS to advance our understanding of galaxy and halo evolution.
Current predictions about the scaling of conformity strength with redshift to $z\sim1$--2, and with halo mass at ${z>0.2}$, cannot be conclusively tested without data of comparable depth from additional fields.
The upcoming DESI survey will push the limit of deep, wide-field spectroscopy, but only to ${z=0.4}$.
