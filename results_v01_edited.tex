%!TEX root = main.tex

\section{Results}\label{sec:results}

In this section we discuss the importance of matching star forming and quiescent IP 
samples in both stellar mass and redshift, and we present the one and two-halo 
conformity signal in the full matched PRIMUS sample.  We discuss the effects of 
cosmic variance on measures of conformity and the need for jackknife
errors at intermediate redshifts, and we investigate the redshift and stellar mass 
dependence of conformity within the PRIMUS sample.

%\subsection{Neighbor Late-type Fractions}\label{sec:LTfraction}
\subsection{The Effects of Matching Redshift and Stellar Mass on the Conformity Signal}\label{sec:LTfraction}

As discussed above, galactic conformity is the observed tendency of neighbor 
galaxies to have the same star-formation type (quiescent/early or star-forming/late) 
as their associated IP galaxy.
One-halo conformity refers to conformity between an IP and the neighbors within the 
same dark matter halo (i.e.~within $\sim$1~Mpc of the IP),
while two-halo conformity refers to conformity between an IP and neighbors in other 
adjacent haloes (i.e.~at distances greater than $\sim$1~Mpc from the IP).

We therefore want to measure how the fraction of neighbors that are late-type differs 
between star-forming and quiescent IP hosts as a function of projected radius from 
the IP.
To do this, for each IP in our matched sample we count all neighbors within 
concentric cylindrical shells of length {$2\times2\,\sigma_{z}(1+z_{\text{IP}})$} 
and cross-sectional area
${\pi[(\Rproj+d\Rproj)^2-\Rproj^2]}$, where $\Rproj$ is the 2D projected radius from the IP in (physical) Mpc, and $d\Rproj$ is the shell width in Mpc.
The late-type fraction of neighbors of star-forming IPs in a cylindrical shell 
at projected radius $\Rproj$ to ${(\Rproj+d\Rproj)}$, $f^{\textrm{SF-IP}}_{\textrm{late}}
(\Rproj)$ is defined to be the sum of the targeting weights (see~\S\ref{sec:targ_weight}) of the late-type neighbors of star-forming IPs in the shell, 
divided by the sum of the
targeting weights of \emph{all} neighbors of star-forming IPs in the shell:
\begin{equation}
        f^{\textrm{SF-IP}}_{\textrm{late}}(\Rproj) = \frac
        {\displaystyle \sum_{i=1}^{N_{\textrm{SF-IP}}} \sum_{j=1}^{N_{\textrm{late},i}} w_{ij} }
        {\displaystyle \sum_{i=1}^{N_{\textrm{SF-IP}}} \sum_{k=1}^{N_{\textrm{tot},i}} w_{ik} }, \nonumber
\end{equation}
and likewise for quiescent IPs.
$N_{\textrm{SF-IP}}$ is the total number of star-forming IPs, $N_{\textrm{late},i}$ is the number of late-type neighbors of IP $i$ in the shell, $N_{\textrm{tot},i}$ is the
total number of neighbors of IP $i$ in the shell, and $w_{ij}$ and $w_{ik}$ are PRIMUS targeting weights of the neighbors.
We are therefore essentially computing neighbor late-type fractions for star-forming 
and quiescent IPs by stacking the neighbors of all IPs of each type.

\begin{figure*}
  \epsscale{1.0}
  \epstrim{0.1in 0.3in 0.4in 0.8in}
%  \fbox{\plotone{figures/unmatchedIPsampleCompare}}
  \plotone{figures/unmatchedIPsampleCompare}
  \caption{
The fraction of late-type neighbor galaxies around star-forming 
and quiescent IPs, to a projected distance of {$\Rproj<15$}~Mpc, 
 for four different IP samples: 
(a)~all IP candidates above the M13 mass completeness limit (\S\ref{sec:mass_limit});
(b)~IP candidates that also have the same redshift distribution for the star-forming
and quiescent IPs; 
(c)~IP candidates that have the same stellar mass distribution;
(d)~IPs that have both matched stellar mass and redshift distributions.
The median redshift and stellar mass of each IP sample are shown in each panel.
}
  \label{fig:IPsample_compare}
\end{figure*}


The importance of matching both the stellar mass and redshift distributions of our 
IP sample is clearly illustrated in Figure~\ref{fig:IPsample_compare}, which shows 
how the late-type fractions of neighbors around star-forming and quiescent IPs 
differ when different IP samples are used.
Figure~\ref{fig:IPsample_compare} shows the fraction of neighbors of star-forming 
and quiescent IPs that are late-type as a function of projected radius from the 
IP in 1~Mpc annuli out to 15~Mpc for four different IP samples.
%

In panel (a) all IP candidates above the M13 mass completeness limit 
(\S\ref{sec:mass_limit}) are included.
Here the median stellar mass of the quiescent IP population is 0.42~dex greater 
than that of the star-forming IP population, and the median redshift is greater 
by 0.05.
This difference in the stellar mass distribution in particular means that 
star-forming IPs are preferentially located in a region of 
Figure~\ref{fig:IPsample_matched} where the PRIMUS sample of
early-type galaxies is incomplete, causing us to overestimate the neighbor late-type fraction for star-forming IPs (at all projected radii).
{\bf(how can that be right, when we are above the completeness limits?  Isn't it 
just that they have different stellar masses and redshifts, and late-type fraction is a function of stellar mass 
and redshift? Clarify this here - should also quote how much the late-type fraction of all PRIMUS galaxies changes from our lowest to our highest redshifts - you made that plot, it was a few percent )}
The result is a relatively fixed offset between the solid and dashed lines in the 
left panel of Figure~\ref{fig:IPsample_compare} that persists to the largest 
projected radii we measure with PRIMUS, mimicking a conformity signal.  We therefore
measure a ``false'' conformity signal in this sample.

In panel (b) we select star-forming and quiescent IP samples with matched redshift 
distributions using the method described in \S\ref{sec:IPsample_matching}.  This 
causes the large-scale offset to disappear, but there is still a difference in the 
median stellar masses of the IP samples of 0.3~dex.  Since late-type fraction depends on stellar mass, that is not ideal. 
In panel (c) we select star-forming and quiescent IP samples with matched stellar mass distributions; this results in a star-forming IP sample with a higher median 
redshift than that of the quiescent IP sample (by 0.05).
In this case the systematic bias mimics the opposite of a conformity signal; the solid line moves closer to the dashed line at all projected radii, actually dropping below it at $>$5~Mpc.

Finally, panel (d) shows results for our matched stellar mass and matched redshift IP sample.
Failure to control for differences in the stellar mass and/or redshift distributions 
can introduce bias into the relative neighbor late-type fractions of star-forming and quiescent IPs.
Only by matching both the stellar mass and redshift distributions of our star-forming and quiescent IP samples do we eliminate systematic biases in neighbor late-type fraction
measurements that could masquerade as a conformity signal.
For the remainer of the paper, the IP samples matched in both stellar mass and redshift are referred to as the ``full'' sample.


\subsection{One and Two-Halo Conformity Signal in Matched Sample}\label{sec:signal}

\begin{figure}
  \epsscale{1.1}
  \epstrim{0.4in 0.1in 0.3in 0.4in}
%  \fbox{\plotone{figures/latefracplot_BSE_IPmatchFBF_PHI37_allz_250kpc}}
  \plotone{figures/latefracplot_BSE_IPmatchFBF_PHI37_allz_250kpc}
  \caption{
The fraction of late-type neighbor galaxies around star-forming 
and quiescent IPs, to a projected distance of {$\Rproj<5$}~Mpc, 
for IP samples matched in both stellar mass and redshift, using 
finer {$d\Rproj=0.25$~Mpc} radial bins for all star-forming (blue solid line) and quiescent (red dashed line) IPs in the full sample.
The errors shown have been computed using bootstrap resampling.
}
  \label{fig:latefrac_full}
\end{figure}


Stacked neighbor late-type fractions for the full sample of star-forming and quiescent IPs are shown in Figure~\ref{fig:latefrac_full}, here using finer radial bins.  The errors in each radial bin here and above have been estimated by bootstrap 
resampling. {\bf(have you defined what this is and how you do it?  If not, give that info here and say that below you will discuss bootstrap vs jackknife errors)}
{\bf(need to describe here what you see in Fig 6, specifically the scale-dependence of both lines and the apparent projected scale at which the one- and two-halo terms dominate. also emphasize that it appears by eye that the two-halo signal may persist to 3 Mpc in our sample)}



%%%%%%% EQUAL WEIGHT TO EACH IP:

\begin{figure}
  \epsscale{1.1}
  \epstrim{0.5in 0.1in 0.3in 0.3in}
%  \fbox{\plotone{figures/latefracplot_BSE_IPmatchFBF_PHI37_allz_median_quartiles}}
  \plotone{figures/latefracplot_BSE_IPmatchFBF_PHI37_allz_median_quartiles}
  \caption{
Similar to Figure~\ref{fig:latefrac_full}, except here $\flate$ is the median of 
the distribution of non-zero individual neighbor late-type fractions for each IP type as a function of $\Rproj$.  This effectively gives equal weight to each IP, instead of upweighting the IPs with more neighbors, as shown in Figure 6.
Also shown here is the mean of the non-zero neighbor late-type fraction distributions of star-forming and quiescent IPs (purple dot-dashed and magenta dotted lines),
and the interquartile range of the combined distribution for both IP types (gray shaded region).
}
  \label{fig:latefrac_quartiles}
\end{figure}


Within a particular radial bin (or shell around each IP), this stacking method 
weights IPs with more neighbors more heavily than those with fewer or no neighbors,
for that bin.
To assess whether this will bias our results, we recompute the late-type fraction, 
now assigning equal weight to each IP by computing the neighbor late-type fraction 
individually for each IP and then taking the median of the distribution of all 
non-zero fractions for both IP types, within each radial bin.
The result is shown in Figure~\ref{fig:latefrac_quartiles}, which also shows the 
mean individual neighbor late-type fraction for both IP types in each radial bin (again using
only non-zero fractions), and the interquartile range of the combined distribution for both IP types. {\bf(walk the reader through this figure - what exactly does the grey region show?  why does it not go to smaller scales?  why do the purple and pink lines not go to smaller scales?  Note the broad range of the inner quartiles - the fraction goes from around 65\% to 90\%.)}
The difference between neighbor late-type fractions for star-forming and quiescent IPs is comparable for equal weighting of IPs as shown here and when each IP is weighted
proportionally to its number of neighbors, as shown in Figure 6.
{\bf(again point out here that there appears to be a difference in the two lines - using either the median or mean - to 3 Mpc)}


\setlength{\tabcolsep}{0.03in}
%\begin{deluxetable*}{cccccccccccc}[!h]
\begin{deluxetable*}{ccrrrcccrcccrcc}
\tabletypesize{large}
\tablecaption{Conformity Signal (Jackknife Errors)\label{table:signal}}
\tablewidth{0pt}
\tablehead{
\multicolumn{2}{c}{$N_{\textrm{IP}}$} & \colhead{} & \colhead{} &
\multicolumn{3}{c}{$0.0 < R < 1.0$~Mpc} & {} &
\multicolumn{3}{c}{$1.0 < R < 3.0$~Mpc} & {} &
\multicolumn{3}{c}{$3.0 < R < 5.0$~Mpc} \\
\cline{1-2}\cline{5-7}\cline{9-11}\cline{13-15} \\
\colhead{SF} & \colhead{Q} &
\multicolumn{1}{c}{$z$} &
\multicolumn{1}{c}{$\log\,(\mstar/\msun)$} &
\colhead{$\signorm$} & \colhead{$\sigma_{\textrm{JK}}$} & \colhead{($\sigma_{\textrm{BS}}$)} & {} &
\colhead{$\signorm$} & \colhead{$\sigma_{\textrm{JK}}$} & \colhead{($\sigma_{\textrm{BS}}$)} & {} &
\colhead{$\signorm$} & \colhead{$\sigma_{\textrm{JK}}$} & \colhead{($\sigma_{\textrm{BS}}$)} \\
\cline{1-15} \\
\multicolumn{15}{c}{Full Sample}
}
\startdata
$4,185$ &
$6,197$ &
[0.20, 1.00] &
[9.13, 11.33] &
$0.053\pm0.015$ & 3.6 & $(6.8)$  & {} &
$0.009\pm0.004$ & 2.5 & $(3.9)$  & {} &
$-0.003\pm0.004$ & 0.7 & $(1.3)$  \\
\cutinhead{Redshift Bins}
$2,241$ &
$3,096$ &
[0.20, 0.59] &
[9.13, 11.25] &
$0.052\pm0.013$ & 4.0 & $(4.9)$  & {} &
$0.007\pm0.004$ & 1.9 & $(2.3)$  & {} &
$-0.005\pm0.006$ & 0.9 & $(1.8)$  \\
$1,945$ &
$3,101$ &
[0.59, 1.00] &
[10.11, 11.33] &
$0.056\pm0.026$ & 2.1 & $(4.5)$  & {} &
$0.014\pm0.007$ & 2.0 & $(3.6)$  & {} &
$0.002\pm0.006$ & 0.4 & $(0.6)$  \\
\cline{1-15} \\
$1,520$ &
$2,047$ &
[0.20, 0.48] &
[9.13, 11.25] &
$0.061\pm0.012$ & 5.1 & $(5.1)$  & {} &
$0.009\pm0.005$ & 1.7 & $(2.4)$  & {} &
$-0.005\pm0.007$ & 0.8 & $(1.4)$  \\
$1,406$ &
$2,086$ &
[0.48, 0.68] &
[9.92, 11.28] &
$0.043\pm0.025$ & 1.7 & $(3.0)$  & {} &
$0.001\pm0.006$ & 0.2 & $(0.3)$  & {} &
$-0.002\pm0.005$ & 0.5 & $(0.7)$  \\
$1,261$ &
$2,064$ &
[0.68, 1.00] &
[10.31, 11.33] &
$0.048\pm0.041$ & 1.2 & $(2.7)$  & {} &
$0.023\pm0.010$ & 2.2 & $(4.2)$  & {} &
$0.004\pm0.008$ & 0.5 & $(0.7)$  \\
\cutinhead{Stellar Mass Bins}
$2,385$ &
$3,069$ &
[0.20, 1.00] &
[9.13, 10.82] &
$0.039\pm0.013$ & 2.9 & $(3.7)$  & {} &
$0.005\pm0.004$ & 1.2 & $(1.6)$  & {} &
$0.000\pm0.004$ & 0.1 & $(0.1)$  \\
$1,801$ &
$3,128$ &
[0.20, 1.00] &
[10.82, 11.33] &
$0.070\pm0.021$ & 3.3 & $(5.9)$  & {} &
$0.014\pm0.004$ & 3.3 & $(3.9)$  & {} &
$-0.008\pm0.006$ & 1.3 & $(2.2)$  \\
\cline{1-15} \\
$1,649$ &
$2,064$ &
[0.20, 0.80] &
[9.13, 10.67] &
$0.051\pm0.015$ & 3.4 & $(3.8)$  & {} &
$0.002\pm0.005$ & 0.3 & $(0.5)$  & {} &
$-0.004\pm0.005$ & 0.8 & $(1.0)$  \\
$1,410$ &
$2,019$ &
[0.20, 1.00] &
[10.67, 10.95] &
$0.024\pm0.019$ & 1.3 & $(1.9)$  & {} &
$0.013\pm0.005$ & 2.5 & $(3.0)$  & {} &
$-0.002\pm0.004$ & 0.6 & $(0.6)$  \\
$1,126$ &
$2,113$ &
[0.21, 1.00] &
[10.95, 11.33] &
$0.089\pm0.020$ & 4.4 & $(5.9)$  & {} &
$0.015\pm0.007$ & 2.3 & $(3.4)$  & {} &
$-0.004\pm0.007$ & 0.6 & $(0.9)$  \\
\enddata
\end{deluxetable*}



%%%%%%%%%  NORMALIZED CONFORMITY SIGNAL:

To better quantify our results we define the normalized conformity signal, 
$\signorm$, at a projected radius of $\Rproj$ as the difference of the neighbor late-type fractions of star-forming and quiescent IPs, 
divided by the mean of these two fractions:

\begin{equation}
	\signorm(\Rproj) = \frac
	{ f^{\textrm{SF-IP}}_{\textrm{late}}-f^{\textrm{Q-IP}}_{\textrm{late}} }
	{ \left( f^{\textrm{SF-IP}}_{\textrm{late}}+f^{\textrm{Q-IP}}_{\textrm{late}} \right) /2}
\end{equation}

Table~\ref{table:signal} presents the 
normalized conformity signal in our full matched sample in integrated radial bins of 
$\Rproj=0$--1, 1--3, and 3--5~Mpc.
Over the full redshift range {$0.2<z<1.0$} we find a normalized 1-halo conformity 
signal of 5.3\% and a 2-halo signal of 1.1\%.
{\bf(emphasize here how small of a signal this is, esp. in the two halo regime! emphasize also how wide of a redshift range you're using and how large of a parent sample was used to get these numbers - would be good to essentially quote the volume used and the observed density of targets used to get this)}




\begin{figure}
  \epsscale{1.1}
  \epstrim{0.3in 0.1in 0.2in 0.3in}
%  \fbox{\plotone{figures/normsigplot_allz_errorCompare}}
  \plotone{figures/normsigplot_allz_errorCompare}
  \caption{Normalized conformity signal, $\signorm$, for the full matched sample measured to {$\Rproj<5$~Mpc}, with both bootstrap (orange) and jackknife errors (black) shown.  The jackknife errors exceed the bootstrap errors by up to a factor of $\sim$2.
}
  \label{fig:normsig_full}
\end{figure}



\subsection{Bootstrap Versus Jackknife Errors}\label{sec:errors}

In Table~\ref{table:signal} we estimate the uncertainty in $\signorm$ using both 
bootstrap and jackknife resampling, and quote the siginificance we find using each 
method as $\sigmaBS$ and $\sigmaJK$, respectively.
We compute bootstrap errors by selecting 90\% of the data randomly with 
replacement 200 times, and then taking the standard deviation of the 200 results.
To compute jackknife errors we divide the survey area of our full IP sample into 10 regions of approximately $0.5~\degsq$ each.
We then compute $\signorm$ 10 times, systematically excluding one of the 10 jackknife samples each time, and take the standard deviation of the 10 results.
{\bf(need to say more about why you do this both ways - what is each way basically giving you information on?)}

Figure~\ref{fig:normsig_full} shows $\signorm$ for the full sample in {$d\Rproj=1$~Mpc} bins with both jackknife and bootstrap errors.
It is important to note that here the jackknife method yields errors that are at least as large as the bootstrap errors and that usually exceed bootstrap errors by a factor of $\sim$2, with the result that $\sigmaJK$ is significantly less than $\sigmaBS$ in all projected radial ranges we test.
In the full sample we find that for {$0<\Rproj<1$~Mpc} the bootstrap error in $\signorm$ is ${\pm0.008}$, which yields a significance of $\sigmaBS=6.8$, while the jackknife error is ${\pm0.015}$, with a significance of $\sigmaJK=3.6$.


% this gets into quartiles results - testing whether the trend is higher w/ larger differences in sSFR 
{\bf(need to motivate why you're doing this test -trying to see if you can get a more significant result, by seeing if there's a stronger signal using the more extreme ends of the distribution)}
We also compute $\signorm$ using only the highest and lowest quartiles of IP specific SFR (also matched in stellar mass and redshift distribution).
The 1-halo term over the full redshift range increases slightly to 5.5\%, while the uncertainty decreases to 1.2\%.
This increases $\sigmaJK$ to 4.7, even though the sample is half the size of the full matched IP sample.
The 2-halo term increases slightly to 1.5\%, but the uncertainty also increases to 0.9\%, which decreases $\sigmaJK$ to from 2.5 to 1.7.






\subsection{Cosmic Variance}\label{sec:cosmic_var}

\begin{figure}
  \epsscale{1.1}
  \epstrim{0.4in 0.7in 0.3in 0.3in}
%  \fbox{\plotone{figures/normsigplot_byField_1halo}}
  \plotone{figures/normsigplot_byField_1halo}
  \caption{
1-halo term of $\signorm$ for each field and the full sample in the range {$0<\Rproj<1$~Mpc}.
Field errors are estimated by bootstrap resampling within the field, while the error on the full sample point is estimated by jackknife resampling.
}
  \label{fig:normsig_fields_1halo}
\end{figure}

\begin{deluxetable}{lrrr}
\tabletypesize{large}
\tablecaption{Signifcance of 1-halo conformity signal ($0<\Rproj<1$~Mpc) for individual fields.
\label{table:signal_fields}}
\tablewidth{0pt}
\tablehead{
\colhead{Field} & \colhead{$N_{\textrm{SF-IP}}$} & \colhead{$N_{\textrm{Q-IP}}$} & \colhead{$\sigma_{\textrm{BS}}$} \\
}
\startdata
CDFS &
$1,139$ &
$1,698$ &
$4.0$ \\
COSMOS &
$731$ &
$1,099$ &
$3.1$ \\
ES1 &
$390$ &
$621$ &
$5.9$ \\
XMM-CFHTLS &
$1,325$ &
$1,897$ &
$3.4$ \\
XMM-SXDS &
$600$ &
$882$ &
$0.2$ \\
\cline{1-4} \\
Full Sample ($\sigma_{\textrm{JK}}$) &
$4,185$ &
$6,207$ &
$3.6$ \\
\enddata
\end{deluxetable}



Errors estimated with jackknife resampling incorporate differences in the 
magnitude of the conformity signal among spatially distinct regions of the sky.
The fact that $\sigmaJK$ is significantly less than $\sigmaBS$ for every conformity signal measurement in Table~\ref{table:signal}
illustrates the importance of accounting for cosmic variance in any conformity measurement.
We further investigate how the conformity signal in PRIMUS is sensitive to cosmic variance by measuring the one-halo term of $\signorm$ {($0<\Rproj<1$~Mpc)} for each field individually.
The results are shown in Table~\ref{table:signal_fields} and Figure~\ref{fig:normsig_fields_1halo}.
We find substantial field-to-field variation within the PRIMUS survey.  
Among the five fields in our full sample, we find that the one-halo term of 
$\signorm$ varies from over $12\%$ with $\sigmaBS=5.9$ in ES1, to $\sim$5\% in 
CDFS, COSMOS, and XMM-CFHTLS, to $0\%$ with $\sigmaBS\simeq0$ in XMM-SXDS.
{\bf(explain what the error bars here are, how they are derived, and what they mean.  also need a final sentence saying something about how this variation between fields celarly indicates the importance of measuring conformity in multiple field, as any one field can be fairly different from the mean - this basically shows that there is a large dispersion in the amount of conformity between fields, each of size one to a few sq. deg.)}



\subsection{Redshift and Stellar Mass Dependence}\label{sec:z_mass_bins}
{\bf(throughout use one-halo and not 1-halo - make consistent )}

\begin{figure*}
  \epsscale{1.0}
  \epstrim{0.2in 0.4in 0.4in 0.8in}
%  \fbox{\plotone{figures/latefrac_normsig_binnedCompare}}
  \plotone{figures/latefrac_normsig_binnedCompare}
  \caption{
Top panels: Neighbor late-type fractions for star-forming (blue solid and dash-dot lines) and quiescent (red dashed lines) IPs in our matched sample divided into two redshift bins (left) and two stellar mass bins (right).  Errors are from bootstrap resampling.
Bottom panels: $\signorm$ for the corresponding redshift and stellar mass divisions in the top panels.  Errors are computed from jackknife resampling.
The bottom panels also show $\signorm$ for the higher redshift bin (left) and higher stellar mass bin (right) computed \emph{without} the COSMOS field (gray dashed lines).
}
  \label{fig:latefrac_normsig_compare}
\end{figure*}


{\bf(motivate why you're doing this - mention the Hearin paper here again (it will be in the introduction as well, so just briefly remind the reader of it here))}
We further divide our full sample into two redshift bins and two stellar mass bins, 
to investigate and dependence in the magnitude of the signal on redshift or 
stellar mass.
In Figure~\ref{fig:latefrac_normsig_compare} we divide the full IP sample into two redshift bins, {$z=0.20$--0.59} and 0.59--1.0, and two stellar mass bins, 
{$\log\,(\mstar/\msun)=9.13$--10.82} and 10.82--11.33, each containing equal numbers of IPs.
The upper panels show $\flate$ for star-forming and quiescent IPs in each redshift or stellar mass bin, while the lower panels plot the corresponding values of
$\signorm$ for each radial bin. The normalized signal and significance are given in 
Table~\ref{table:signal}.

When dividing into redshift bins the 1-halo term of $\signorm$ in both bins is comparable to the 5.3\% signal observed over the full redshift range.
The significance of the ``low'' redshift {($z=0.20$--0.59)} 1-halo term increases to 
{$\sigmaJK=4.0$}, while the significance of the ``high'' redshift 1-halo term drops to
{$\sigmaJK=2.1$}.
The magnitude of the 2-halo term of $\signorm$ in each bin also remains comparable to the full redshift range signal of 1.1\%, but the uncertainty in each bin also increases,
reducing $\sigmaJK$ from 2.5 for the full redshift range to 1.7 and 1.6 for the low and high redshift bins, respectively.
{\bf(what is the upshot?)}

{\bf(throughout use signal($\pm$error))}
When dividing the full sample into two stellar mass bins, the signal in the 
1-halo term drops to {$3.9(\pm1.3)$\%} ($\sigmaJK=2.9$) for the low stellar mass bin ({$\log\,(\mstar/\msun)=9.13$--10.82}) but
increases to {$(7.0\pm2.1)$\%} ($\sigmaJK=3.3$) for the high mass bin {$(\log\,(\mstar/\msun)=10.82$--11.33)}.
Similarly, the signal of the 2-halo term decreases for the low stellar mass bin 
and increases for the high stellar mass bin, although $\sigmaJK$ is only 1.5 
and 2.8 for these stellar mass bins, respectively.  This result of a 
stronger conformity signal for higher stellar mass would disagree with 
\todo{...}, but we emphasize that the low values of 
$\sigmaJK$ obtained when dividing the full IP sample into even two bins 
make it difficult to draw significant conclusions about the presence of any 
stellar mass or redshift
dependence of the conformity signal in PRIMUS.  This is especially true 
for the 2-halo term.




{\bf(probably want another subsection with any plots we want to show from Behroozi discussion)}

\begin{figure}
  \epsscale{1.1}
  \epstrim{0.45in 0.1in 0.15in 0.3in}
%  \fbox{\plotone{figures/IPlatefrac_vs_environ}}
  \plotone{figures/IPlatefrac_vs_environ}
  \caption{Late-type fraction of IPs as a function of (weighted) number of neighbors within {$0.3<\Rproj<4$~Mpc}, where the minimum neighbor mass is $\mIP-0.5$~dex.
Results are shown for $\log\,(\mIP/\msun)$={10.0--10.5} (magenta triangles) and {10.5--11.0} (blue diamonds).
The errors shown are computed from jackknife resampling.
}
  \label{fig:latefrac_vs_environ}
\end{figure}



%\subsection{COSMOS}\label{sec:cosmos}




%\begin{figure}
%  \epsscale{1.1}
%  \epstrim{0.3in 0.6in 0.4in 1in}
%  \fbox{\plotone{figures/normsigplot_JKE_IPmatchFBF_PHI37_panels}}
%  \plotone{figures/normsigplot_JKE_IPmatchFBF_PHI37_panels}
%  \caption{
%Top panel:
%\todo{replace with previous figure?}
%Normalized conformity signal for the full sample to $\Rproj=5$~Mpc with jackknife errors.
%The signal is also shown excluding the COSMOS field (gray dashed line).
%Middle panel:
%Normalized conformity signal for two redshift bins containing equal numbers of IP galaxies (jackknife errors).
%The signal for the higher redshift bin is also shown excluding the COSMOS field (gray dashed line).
%Bottom panel:
%Normalized conformity signal for two stellar mass bins containing equal numbers of IP galaxies (jackknife errors).
%The signal for the greater stellar mass bin is also shown excluding the COSMOS field (gray dashed line).
%}
%  \label{fig:normsig_zbins_massbins}
%\end{figure}

%\begin{figure}
%  \epsscale{1.1}
%  \epstrim{0.3in 0.1in 0.3in 0.4in}
%  \fbox{\plotone{figures/normsigplot_byField}}
%  \plotone{figures/normsigplot_byField}
%  \caption{Normalized conformity signal for each field with bootstrap errors.
%\todo{probably too busy; just have following figure?}
%}
%  \label{fig:normsig_fields}
%\end{figure}

%\begin{figure}
%  \epsscale{1.1}
%  \epstrim{0.2in 0.2in 0.4in 0.4in}
%  \fbox{\plotone{figures/normsigplot_JKE_IPmatchFBF_PHI37_zbinsEqualHalves}}
%  \plotone{figures/normsigplot_JKE_IPmatchFBF_PHI37_zbinsEqualHalves}
%  \caption{Normalized conformity signal for three redshift bins containing equal numbers of IP galaxies.
%}
%  \label{fig:normsig_zbins}
%\end{figure}

%\begin{figure}
%  \epsscale{1.1}
%  \epstrim{0.2in 0.2in 0.4in 0.4in}
%  \fbox{\plotone{figures/normsigplot_JKE_IPmatchFBF_PHI37_massbinsEqualHalves}}
%  \plotone{figures/normsigplot_JKE_IPmatchFBF_PHI37_massbinsEqualHalves}
%  \caption{Same as Figure~\ref{fig:normsig_zbins} except the IP samples have instead been divided into three stellar mass bins containing equal numbers of IP galaxies.
%}
%  \label{fig:normsig_massbins}
%\end{figure}

%\begin{figure}
%  \epsscale{1.1}
%  \epstrim{0.2in 0.2in 0.4in 0.4in}
%  \fbox{\plotone{figures/normsigplot_JKE_IPmatchFBFnoCosmos_PHI37_allz}}
%  \plotone{figures/normsigplot_JKE_IPmatchFBFnoCosmos_PHI37_allz}
%  \caption{Same as Figure~\ref{fig:normsig_all} except the COSMOS field has been excluded.
%}
%  \label{fig:}
%\end{figure}

%\subsubsection{Redshift Bins}\label{sec:cosmos_zbins}

%\begin{figure}
%  \epsscale{1.1}
%  \epstrim{0.2in 0.2in 0.4in 0.4in}
%  \fbox{\plotone{figures/normsigplot_JKE_IPmatchFBFnoCosmos_PHI37_zbinsEqualThirds}}
%  \plotone{figures/normsigplot_JKE_IPmatchFBFnoCosmos_PHI37_zbinsEqualThirds}
%  \caption{Same as Figure~\ref{fig:normsig_zbins} except the COSMOS field has been excluded.
%}
%  \label{fig:}
%\end{figure}

%\begin{figure}
%  \epsscale{1.1}
%  \epstrim{0.4in 0.2in 0.4in 0.4in}
%  \fbox{\plotone{figures/latefracplot_BSE_IPmatchFBF_COSMOS_PHI37_allz}}
%  \plotone{figures/latefracplot_BSE_IPmatchFBF_COSMOS_PHI37_allz}
%  \caption{Late-type neighbor fractions at $\Rproj \le 15$~Mpc for all late- (solid blue) and early-type (dashed red) IPs in the COSMOS field only.
%}
%  \label{fig:latefrac_COSMOS}
%\end{figure}

%\subsubsection{Stellar Mass Bins}\label{sec:cosmos_massbins}

%\begin{figure}
%  \epsscale{1.1}
%  \plotone{figures/normsigplot_JKE_IPmatchFBFnoCosmos_PHI37_massThirdsEqualIP}
%  \caption{
%}
%  \label{fig:}
%\end{figure}
