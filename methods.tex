%!TEX root = main.tex

\section{Methods}\label{sec:methods}

\subsection{Star-Forming vs.~Quiescent}\label{sec:SFvsQ}

Each galaxy our the sample was classified as star-forming (late-type) or quiescent (early-type) using the following equation \todo{source!}:

\begin{equation}\label{eq:}
	\log(\text{SFR}) = -1.29 + 0.65 \log(\mstar-10) + 1.33(z-0.1) %\nonumber
\end{equation}

where SFR is in units of $\msun$~yr$^{-1}$ and $\mstar$ is stellar mass in units of $\msun$.

\subsection{Isolated Primaries}\label{sec:IP}

No group catalog

Similar criteria as Kauffman et al.~(2013)

A galaxy is classified as an isolate primary (IP) if there are no other galaxies with stellar mass greater than $\mstar/2$ within a projected radius $\Rproj$ and redshift difference $\Delta z$ from the IP candidate.

$\Delta z$ chosen to integrate over peculiar velocities and redshift distortions: PRIMUS redshift resolution is limited by $\Delta z=0.005(1+z)$.

Because PRIMUS is a flux-limited survey, the minimum stellar mass of a galaxy at which the sample is complete increases with redshift.  It also depends on field and galaxy star-formation status.  To avoid falsely classifying a galaxy as an isolated primary, a galaxy was only considered to be a candidate for IP status if its stellar mass is greater than the stellar mass at which our sample is complete.  A galaxy with star-formation status $s$ and redshift $z$ in field $f$ was only considered to be an IP candidate if its stellar mass is at least $\mMoustakas(f,\,s,\,z)-0.5$~dex, where $\mMoustakas$ are the stellar mass completeness limits for PRIMUS from \citet{Moustakas13}.  These values are shown in Table~\ref{table:masslimits}.

\subsubsection{Default Parameters}\label{sec:defaultParams}

$\Rproj=500$~kpc

$\Delta z=0.005(1+z_{\text{IP}})$ 

Minimum stellar mass for IP candidate status: $\mstar \ge \mMoustakas-0.5$~dex

At $z\le1.0$ our sample is complete at $\langle \log(\mstar/\msun) \rangle \ge \#$, where the average is taken over all fields.  Of the \totalgals~galaxies in our sample, $\#$ (or $\#\%$) are isolated primaries.

\subsubsection{Other Parameter Sets}

More conservative mass completeness limit: $\mstar \ge \mMoustakas$

Single mass completeness limit independent of field, redshift, and SF status: $\mstar \ge 10^{10} \msun$

Different cylinder depths: $0.005(1+z)/\Delta z=0.5$ to 3.0

%%!TEX root = ../main.tex

\begin{deluxetable}{cccccc}
\tablewidth{0pc}
\tablecolumns{6}
\tablecaption{Stellar Mass Completeness Limits}
\tablehead{
\colhead{} & 
\colhead{COSMOS} & 
\colhead{XMM-SXDS} & 
\colhead{XMM-CFHTLS} & 
\colhead{CDFS} & 
\colhead{ELAIS-S1} \\
\cline{1-6}
\\
\colhead{Redshift Range} & 
\multicolumn{5}{c}{$\log$} 
}
\label{table:masslimits}
\startdata
%\cline{1-6}
$0.20-0.30$ & 
 8.73 &  8.86 &  8.95 &  9.62 &  9.70 \\
$0.30-0.40$ & 
 9.14 &  9.23 &  9.23 &  9.87 &  9.99 \\
$0.40-0.50$ & 
 9.51 &  9.58 &  9.51 & 10.10 & 10.26 \\
$0.50-0.65$ & 
 9.92 &  9.97 &  9.87 & 10.37 & 10.56 \\
$0.65-0.80$ & 
10.33 & 10.38 & 10.31 & 10.65 & 10.87 \\
$0.80-1.00$ & 
10.71 & 10.78 & 10.83 & 10.94 & 11.17 \\
\enddata
\end{deluxetable}

\subsection{Late-Type Fraction}

We next counted the total number and number of star-forming galaxies within $\Delta v$ within $\Rproj$ to $\Rproj~+~0.25$~Mpc around each IP, and added the results for all IPs to obtain the fraction of late-type galaxies for each annulus.  All galaxies were allowed to contribute to the totals for each IP, such that one IP could be counted as a satellite of a different IP.

Figure~\ref{fig:} shows the fraction of galaxies that are late-type as a function of projected radius around early-type and late-type IPs for our default sample.  Error bars are Poissonian.

